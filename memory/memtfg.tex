\documentclass[a4paper,12pt,oneside]{scrbook}
\KOMAoptions{
    chapterprefix=true,
    parskip=half,
    captions=tableheading,
    draft=false
}

\usepackage{iftex}

% Activar números de línea (Descomentar para borradores)
% \usepackage{lineno}
% \linenumbers
% \setlength\linenumbersep{5pt}
% \renewcommand\linenumberfont{\normalfont\tiny\sffamily\color{gray}}

\usepackage[spanish,es-nolists,es-tabla,es-noindentfirst]{babel}
\usepackage[hidelinks]{hyperref}

\renewcommand{\listtablename}{Índice de tablas}    
\renewcommand{\listfigurename}{Índice de figuras}
\newcommand*{\listofloaname}{Índice de algoritmos}

\addto\extrasspanish{
    \renewcommand{\sectionautorefname}{Apartado}
    \renewcommand{\subsectionautorefname}{Apartado}
    \renewcommand{\subsubsectionautorefname}{Apartado}
}

% Márgenes del documento
\usepackage{geometry}
\geometry{left=2cm, right=2cm, top=2cm, bottom=3.3cm}

% Configuración de fuentes
\ifPDFTeX
    \usepackage[T1]{fontenc}
    \usepackage[utf8]{inputenc}
    \usepackage{lmodern}
    \usepackage{amsmath}
    \usepackage{amssymb}
    \usepackage{textcomp}
    \usepackage[scale=0.85]{sourcecodepro}

    \renewcommand{\familydefault}{\sfdefault}
    \newcommand{\lmromanfont}{\rmfamily}
\else % XeLaTeX / LuaLaTeX
    \usepackage{fontspec}
    \usepackage{amsmath}
    \usepackage{unicode-math}   
    
    \setmainfont{Latin Modern Sans}[
        Ligatures=TeX,
        SmallCapsFont=Latin Modern Roman Caps
    ]
    \setsansfont{Latin Modern Sans}[Ligatures=TeX]
    \setmonofont{Source Code Pro}[Scale=MatchLowercase]
    \newfontfamily{\lmromanfont}{Latin Modern Roman}
\fi

% Configuración de listas no numeradas
\setkomafont{itemizelabel}{\lmromanfont}
\renewcommand{\labelitemii}{$\circ$}
\renewcommand{\labelitemiii}{$\diamond$}
\renewcommand{\labelitemiv}{$\triangleright$}

% Citas
\usepackage{csquotes}
\renewcommand{\mkbegdispquote}[2]{«}
\renewcommand{\mkenddispquote}[2]{#1»#2}

%% Bibliografía
\usepackage[style=numeric-comp,sorting=none,backend=biber]{biblatex}
\defbibnote{note}{Las siguientes referencias bibliográficas se presentan en orden de aparición en el texto.\par\bigskip}

\addbibresource{referencias.bib}

% Paquetes adicionales
\usepackage{caption}
\usepackage{enumitem}
\usepackage{float}
\usepackage{graphicx}
\usepackage{lipsum}
\usepackage{microtype}
\usepackage{pdflscape}
\usepackage{rotating}
\usepackage{subcaption}
\usepackage[table]{xcolor}
\usepackage{siunitx}

% --- IMPORTANTE: RUTA DE IMÁGENES ---
\graphicspath{ {../images/} }

\ifPDFTeX
    \newcommand{\perthousand}{\textperthousand}
\else
    \newcommand{\perthousand}{‰}
\fi

% Tablas
\usepackage{array} 
\usepackage{booktabs}
\usepackage{longtable}
\usepackage{multirow}
\usepackage{tabularx}
\newcolumntype{R}{>{\raggedleft\arraybackslash}X}
\newcolumntype{C}{>{\centering\arraybackslash}X}
\newcolumntype{M}{>{$}c<{$}}
\newcolumntype{L}{>{$}l<{$}}
\newcolumntype{N}{>{$}r<{$}}
\renewcommand*{\arraystretch}{1.5}

% Código con coloreado de sintaxis
\usepackage{scrhack}            
\usepackage{listings}
\renewcommand{\lstlistingname}{Listado}
\lstset{
    basicstyle=\ttfamily\small,
    breaklines=true,
    columns=fullflexible,
    keepspaces=true,
    numbers=left,
    numberstyle=\tiny\color{gray},
    keywordstyle=\color{blue},
    commentstyle=\color{gray},
    stringstyle=\color{red}
}

% Descripción de algoritmos en pseudocódigo
\usepackage[chapter]{algorithm}
\usepackage{algpseudocodex}
\floatname{algorithm}{Algoritmo}
\captionsetup[algorithm]{labelfont=bf, labelsep=default}

% Entornos personalizados
\newenvironment{abstract}[1][\abstractname]{
    \cleardoublepageusingstyle{empty}
    \thispagestyle{empty}
    \begin{center}\textbf{#1}\end{center}
    \begin{itshape}\par\noindent%
}
{\end{itshape}}

\newenvironment{keywords}[1][Keywords]
{\vspace{7pt}\par\noindent\textup{\textbf{#1: }}\begin{upshape}}
{\end{upshape}}

\newcommand{\license}[2]{%
\begin{center}
    \begin{minipage}{0.8\textwidth}   
        \begin{center}
            \includegraphics[width=4.66cm]{#1}\\[12pt]
            {\Large #2}
        \end{center}
    \end{minipage}
\end{center}
}

% Formato de capítulos
\renewcommand*{\chapterformat}{%
    \mbox{\chapappifchapterprefix{\nobreakspace}\thechapter
    \IfUsePrefixLine{}{\enskip}}}
\renewcommand{\chapterheadmidvskip}{\vskip 0pt}
\renewcommand*{\figureformat}{\figurename~\thefigure}
\renewcommand*{\tableformat}{\tablename~\thetable}

\let\originaladdchaptertocentry\addchaptertocentry
\renewcommand*{\addchaptertocentry}[2]{%
  \IfArgIsEmpty{#1}{%
    \originaladdchaptertocentry{#1}{#2}%
  }{%
    \originaladdchaptertocentry{}{\chapapp~#1\autodot\space#2}%
  }%
}

\begin{document}   

%%%%%%%%%%%%%%%%%%%%%%%%%%%%%%%%%%%%%%%%%%%%%%%%%%%%%%%%%%%%%%%%%%%%%%%%%%%%%%%
% PORTADA
%%%%%%%%%%%%%%%%%%%%%%%%%%%%%%%%%%%%%%%%%%%%%%%%%%%%%%%%%%%%%%%%%%%%%%%%%%%%%%%

% Metadatos del PDF
\hypersetup{
    pdftitle={Sistema híbrido de Predicción Electoral mediante Análisis de datos abiertos gubernamentales y opinión en Redes Sociales},
    pdfauthor={Mario Guerra Pérez},
    pdfsubject={Trabajo de Fin de Grado},
}

\pagestyle{empty}
\newcommand{\HRule}{\rule{\linewidth}{0.3mm}}
{
    \setlength{\parindent}{0mm}
    \setlength{\parskip}{0mm}
    
    \vspace*{1.20cm}
    \includegraphics[width=9.81cm]{../images/logos/escuela-ingenieria-tecnologia-original} 
    \vspace*{\stretch{0.9}}
    
    {\centering
    \fontsize{32pt}{32pt}\selectfont Trabajo de Fin de Grado\\[10pt]
    \fontsize{20pt}{20pt}\selectfont Grado en Ingeniería Informática\par}
    \HRule\vspace*{-2mm}
    \begin{flushright}
        {\fontsize{26pt}{26pt}\selectfont Sistema híbrido de Predicción Electoral mediante Análisis de datos abiertos gubernamentales y opinión en Redes Sociales\par
        \vspace*{3mm}
        \fontsize{18pt}{18pt}\selectfont \textit{Hybrid Election Prediction System using Analysis of Open Government Data and Social Media Opinion}\par
        \vspace*{11mm}
        \fontsize{16pt}{16pt}\selectfont Mario Guerra Pérez
        }\vspace*{12mm}
    \end{flushright}
    \HRule
    
    \vspace*{\stretch{1}}
    \begin{center}
        \fontsize{18pt}{18pt}\selectfont La Laguna, \today
    \end{center}
}

%%%%%%%%%%%%%%%%%%%%%%%%%%%%%%%%%%%%%%%%%%%%%%%%%%%%%%%%%%
% PRELIMINARES (Certificado, Agradecimientos, Licencia, Resumen)
%%%%%%%%%%%%%%%%%%%%%%%%%%%%%%%%%%%%%%%%%%%%%%%%%%%%%%%%%%
\frontmatter

% AQUÍ ES DONDE CARGAMOS EL OTRO ARCHIVO
% Y YA NO PONEMOS NADA MÁS DEBAJO HASTA EL ÍNDICE
% -------------------------------------------------------------------
% 1. CERTIFICADO DEL TUTOR
% -------------------------------------------------------------------
\cleardoublepageusingstyle{empty}
\thispagestyle{empty}

{
\setlength\parindent{0pt}
    D. \textbf{José Luis González Ávila}, profesor Titular de Universidad adscrito al Departamento de Ingeniería Informática y de Sistemas de la Universidad de La Laguna, como tutor.
    
    \bigskip
    \bigskip
    \textbf{C E R T I F I C A}

    \bigskip
    Que la presente memoria titulada:
    
    \bigskip
    \begin{quote}
    \textit{``Sistema híbrido de Predicción Electoral mediante Análisis de datos abiertos gubernamentales y opinión en Redes Sociales''}
    \end{quote}
    
    \bigskip
    \noindent ha sido realizada bajo su dirección por D. \textbf{Mario Guerra Pérez}.
    
    \bigskip
    Y para que así conste, en cumplimiento de la legislación vigente y a los efectos
    oportunos, firman la presente en La Laguna a \today.
}

% -------------------------------------------------------------------
% 2. AGRADECIMIENTOS
% -------------------------------------------------------------------
\cleardoublepageusingstyle{empty}
\thispagestyle{empty}

\begin{flushright}
    \setlength{\parindent}{0mm}

    {\LARGE Agradecimientos}
    \vspace{15mm}
    
    \begin{large}
        \begin{minipage}{0.6\textwidth} % Ajustado ancho para que quepa mejor el texto
            \begin{flushright}
            A mi familia, por su incondicional apoyo durante estos años de estudio.
            
            A mi tutor, José Luis González Ávila, por su guía y paciencia en la elaboración de este trabajo.
            
            A mis compañeros, por hacer el camino más llevadero.
            \end{flushright}
        \end{minipage}
    \end{large}
\end{flushright}

% -------------------------------------------------------------------
% 3. LICENCIA (BY-NC-ND)
% -------------------------------------------------------------------
\cleardoublepageusingstyle{empty}
\thispagestyle{empty}

{\noindent\LARGE Licencia}
\vspace{15mm}

% Asegúrate de que la imagen 'by-nc-nd.eu' está en la carpeta 'images/licenses/'
% Si la tienes suelta en images, cambia la ruta a: images/by-nc-nd.eu
\license{images/licenses/by-nc-nd.eu}{© Esta obra está bajo una licencia de Creative Commons Reconocimiento-NoComercial-SinObraDerivada 4.0 Internacional.}

% -------------------------------------------------------------------
% 4. RESUMEN (ESPAÑOL)
% -------------------------------------------------------------------
\begin{abstract}
El presente Trabajo de Fin de Grado (TFG) tiene como objetivo desarrollar un sistema de predicción electoral que integre dos fuentes de información heterogéneas: datos históricos estructurados provenientes de fuentes gubernamentales (Open Data) y datos no estructurados de opinión extraídos de redes sociales. En este contexto, se analizará el sentimiento político durante las elecciones generales de España previstas para el año 2027, utilizando como fuente primaria de opinión los mensajes en Twitter (como en Facebook).

Este análisis se fundamenta en el empleo de técnicas de Procesamiento de Lenguaje Natural (PLN) para clasificar las publicaciones que los usuarios hayan realizado en función de las categorías de sentimiento principales: positivo, negativo o neutro. Un componente crítico es la detección de bots o cuentas no humanas que puedan alterar la estadística. El propósito general es determinar si existe una correlación entre el sentimiento predominante en la red social, contrastado con la base histórica, y los resultados electorales reales.

Para realizar dicho análisis, la metodología abarca, por un lado, la obtención de los datasets electorales históricos y, por otro, la recolección de datos de Twitter empleando técnicas de extracción (scraping). Para el filtrado de cuentas corporativas y bots, se utilizarán heurísticas basadas en la ubicación de la cuenta y la fecha de creación del perfil. Una vez adquirida la información, se procesará para eliminar ruido, asegurando la calidad de la muestra.

Finalmente, se compararán los resultados con la realidad en comparación con otros países y se estudiará la correlación entre las interacciones digitales y los mecanismos democráticos.

\begin{keywords}[Palabras clave]
Análisis de Sentimiento, Redes Sociales, Predicción Electoral, PLN, Open Data, Twitter, Detección de Bots, Polaridad Política.
\end{keywords}
\end{abstract}

% -------------------------------------------------------------------
% 5. ABSTRACT (INGLÉS)
% -------------------------------------------------------------------
\cleardoublepageusingstyle{empty}
\thispagestyle{empty}

\begin{abstract}[Abstract]
This Bachelor's Thesis aims to develop an election prediction system integrating two heterogeneous information sources: structured historical data from government sources (Open Data) and unstructured opinion data extracted from social media. In this context, political sentiment during the upcoming Spanish general elections (2027) will be analyzed using Twitter as the primary source.

The analysis relies on Natural Language Processing (NLP) techniques to classify posts based on sentiment (positive, negative, neutral) and to detect bots that might distort statistics. The goal is to determine if there is a correlation between social media sentiment, benchmarked against historical data, and actual election results.

The methodology includes data collection via scraping, filtering based on geolocation and account creation date, and classification using machine learning models to quantify polarity and voting intention.

\begin{keywords}
Sentiment Analysis, Social Media, Election Prediction, NLP, Open Data, Twitter, Bot Detection, Political Polarity.
\end{keywords}
\end{abstract}

%%%%%%%%%%%%%%%%%%%%%%%%%%%%%%%%%%%%%%%%%%%%%%%%%%%%%%%%%
% ÍNDICES
%%%%%%%%%%%%%%%%%%%%%%%%%%%%%%%%%%%%%%%%%%%%%%%%%%%%%%%%%
\pagestyle{plain}
\cleardoublepage
\setcounter{page}{1} 

\tableofcontents
\listoffigures
\listoftables
%\listofalgorithms

%%%%%%%%%%%%%%%%%%%%%%%%%%%%%%%%%%%%%%%%%%%%%%%%%%%%%%%%%%
% CONTENIDOS (CAPÍTULOS)
%%%%%%%%%%%%%%%%%%%%%%%%%%%%%%%%%%%%%%%%%%%%%%%%%%%%%%%%%%
\mainmatter

% NOTA: Asegúrate de que estos archivos .tex están dentro de la carpeta "chapters/"
\chapter{Introducción}
\label{ch:introducción}

Cualquier capítulo puede tener múltiples apartados, como el \autoref{sec:lista_de_items} o el \autoref{sec:enumeraciones} de este mismo capítulo.

También está el \autoref{sec:primera_sección} del \autoref{ch:capítulo-dos} que tiene la \autoref{fig:otra}.

Se puede utilizar \verb|\indent| o \verb|\noindent| al principio de un párrafo para añadir o eliminar la sangría en el párrafo, respectivamente.

\section{Listas de elementos}
\label{sec:lista_de_items}

Esta es la lista de elementos del \autoref{sec:lista_de_items}:

\begin{itemize}
    \item Item 1
    \begin{itemize}
        \item Item 1
        \item Item 2
        \item Item 3
        \item Item 4
    \end{itemize}
    
    \item Item 2
    \item Item 3
    \item Item 4
\end{itemize}

\section{Enumeraciones}
\label{sec:enumeraciones}

Esto es una lista enumerada, que puede estar relacionada con la \autoref{fig:introducción}:

\begin{enumerate}
    \item Item 1
    \begin{enumerate}
        \item Item 1
        \item Item 2
        \item Item 3
    \end{enumerate}
    \item Item 2
    \item Item 3
\end{enumerate}

\section{Figuras y tablas}

En la \autoref{fig:introducción} se puede ver una figura de ejemplo. Las tablas, las figuras y los algoritmos (ver el \autoref{sec:algoritmo-zzz}) son flotantes. Esto quiere decir que \LaTeX{} los intentará ubicar en el mejor lugar posible al componer el documento, intentando respetar ciertas reglas tipográficas. Como este lugar puede ser diferente a la posición que realmente ocupan en el texto, \textbf{es importante referenciar en el texto todas las figuras y las tablas}, en los diferentes puntos donde se hable de ellas.

\begin{figure}[htbp]
   \centering
   \includegraphics[width=0.8\linewidth]{images/figura_1}
   \caption{Ejemplo de figura.}
   \label{fig:introducción}
\end{figure}

La \autoref{tbl:presupuesto} en el \autoref{ch:presupuesto} es un ejemplo de tabla hecha con el paquete \verb|tabularx|.

Al crear tablas, figuras u otros elementos flotantes es aconsejable indicar siempre los especificadores de ubicación \verb|[htbp]|, tal y como se hace en los ejemplos de esta plantilla. De esta forma \LaTeX{} intentará primero ponerlos en el lugar; si no puede, intentará ponerlos en la parte superior o inferior de la misma página y, en caso extremo, los pondrá en páginas especiales que solo contienen flotantes.

No es buena idea usar especificadores como \verb|[!h]| o \verb|[H]| para forzar una ubicación determinada. El motivo es que eso impide a \LaTeX{} buscar la mejor forma de componer el documento, pudiendo dar como resultado páginas que se ven muy raras --por ejemplo, dejando muchos huecos libres entre el texto--.
Solo se deben utilizar estos especificadores cuando es absolutamente necesario, como ocurre en el \autoref{ch:presupuesto}, donde interesa que las tablas del presupuesto aparezca juntas, en la posición preestablecida.

\section{Código y algoritmos}

En el \autoref{ch:apéndice-uno} se pueden observar varios ejemplos de entornos para describir algoritmos y código.

\section{Citas}

Las referencias bibliográficas se deben indicar en el archivo \texttt{referencias.bib}, en formato Biblatex \cite{overleaf_biblatex}, y se citan en el texto. Las referencias no citadas no aparecerán en el apartado de la bibliografía.

Las citas pueden ser \emph{en línea} con el texto, como en \cite{examplegithub}, o entre paréntesis \parencite{examplearticle}. Sin embargo, no hay diferencia entre ambos tipos cuando se usa el estilo de cita numérico, que es el estilo por defecto en esta plantilla. Para apreciar la diferencia es necesario activar un estilo de cita como APA.

Las reglas para citar descritas en la guía de la ULL \cite{ulllibguide} permiten citar cualquier cosa: artículos de investigación, libros, entradas de la Wikipedia, blogs, vídeos de Youtube o repositorios de GitHub, entre otros. 

En el \autoref{ch:capítulo-cuatro} se puede ver otro tipo de cita, usando el paquete \texttt{csquotes}, donde se traslada de forma literal una porción del texto original al documento.
 
\section{Otra sección...}

\lipsum[1]

\subsection{Con subsección...}

\lipsum[2]
\chapter{Título del Capítulo 2}
\label{ch:capítulo-dos}

Los capítulos intermedios sirven para cubrir los siguientes aspectos: antecedentes, problemática o estado del arte, objetivos, fases y desarrollo del proyecto.

En el capítulo anterior se ha introducido la \autoref{fig:introducción} y en este la \autoref{fig:otra}. 

\section{Primera sección de otro capítulo}
\label{sec:primera_sección}

\begin{figure}[htbp]
   \centering
   \includegraphics[width=0.5\linewidth]{images/licenses/by-nd}
   \caption{Otra figura.}
   \label{fig:otra}
\end{figure}

\lipsum[3]

\subsection{Primera subsección}

\lipsum[4]

\subsection{Segunda subsección}

\lipsum[5]

\section{Segunda sección de otro capítulo}

\lipsum[6-7]



\chapter{Título del Capítulo 3}
\label{ch:capítulo-tres}

Los capítulos intermedios sirven para cubrir los siguientes aspectos: antecedentes, problemática o estado del arte, objetivos, fases y desarrollo del proyecto.

\section{Primera sección de este capítulo}

\lipsum[-2]

\section{Segundo apartado de este capítulo}

\lipsum[3]

\section{Tercer apartado de este capítulo}

\lipsum[4]
\chapter{Título del Capítulo 4}
\label{ch:capítulo-cuatro}

Los capítulos intermedios sirven para cubrir los siguientes aspectos: antecedentes, problemática o estado del arte, objetivos, fases y desarrollo del proyecto.

En el \autoref{ch:introducción} se comentó lo que \cite{examplearticle} dijo al respecto. Aquí vamos a profundizar en una de sus afirmaciones más controvertidas:

\begin{displayquote}[Albert Einstein]
\lipsum[7]
\end{displayquote}

Es decir que \textquote{erat ac sagittis sempe}, lo que se ilustra en el esquema de la \autoref{fig:otra}.

\lipsum[2]
\chapter{Conclusiones y líneas futuras}
\label{ch:conclusiones}

Este capítulo es obligatorio. Toda memoria de trabajo de fin de grado debe incluir unas conclusiones y unas líneas de trabajo futuro.
\chapter{Summary and Conclusions}
\label{ch:conclusions}

This chapter is compulsory. The memory should include an extended summary and conclusions in English.
\chapter{Presupuesto}
\label{ch:presupuesto}

Este capítulo es obligatorio. Toda memoria de trabajo de fin de grado debe incluir un presupuesto.

\section{Sección Uno}

\begin{table}[!ht]
\centering
\caption{Presupuesto de Equipos y Licencias}
\begin{tabularx}{0.8\textwidth}{Xcr}
    \toprule
    \textbf{Descripción} & \textbf{Cantidad} & \textbf{Coste (€)} \\
    \midrule
    Portátil & 1 & 900,00 \\
    Licencia de Software de Desarrollo (IDE) & 1 & 100,00 \\
    Licencia de Software de Diseño Gráfico & 1 & 50,00 \\
    Compra de Componentes Adicionales & 1 & 150,00 \\
    Servicios en la Nube & 12 meses & 240 \\
    \midrule
    \textbf{Subtotal de Equipos y Licencias} & & \textbf{1440,00} \\
    \bottomrule
\end{tabularx}
\end{table}

\begin{table}[!ht]
\centering
\caption{Coste de Mano de Obra}
\begin{tabularx}{0.8\textwidth}{Xcr}
    \toprule
    \textbf{Descripción} & \textbf{Horas} & \textbf{Coste (€)} \\
    \midrule
    Precio por Hora & & 20,00 \\
    Total de Horas de Trabajo & 100 & \\
    \midrule
    \textbf{Costo Total del Trabajo Humano} & & \textbf{2000,00} \\
    \bottomrule
\end{tabularx}
\end{table}

\begin{table}[!ht]
\centering
\caption{Coste Total del Proyecto}
\label{tbl:presupuesto}
\begin{tabularx}{0,8\textwidth}{Xr}
    \toprule
    \textbf{Descripción} & \textbf{Coste Total (€)} \\
    \midrule
    Subtotal de Equipos y Licencias & 1440,00 \\
    Costo Total del Trabajo & 2000,00 \\
    \midrule
    \textbf{Coste Total del Proyecto} & \textbf{3440,00} \\
    \bottomrule
\end{tabularx}
\end{table}


\appendix

% APÉNDICES
\chapter{Título del Apéndice 1}
\label{ch:apendice-uno}

\section{Algoritmo XXX}
\label{sec:algoritmo-xxx}

Ejemplo de código con coloreado de sintaxis:

\begin{lstlisting}[language=C++]
#include <iostream>

int main()
{
  // Imprime "Hello, world!" en la consola
  std::cout << "Hello, world!\n";
  return 0;
}
\end{lstlisting}

\section{Archivo XXY}
\label{sec:archivo-xxy}

Ejemplo de JSON usando el mismo entorno de coloreado de sintaxis:

\begin{lstlisting}[language=json]
{
    "nombre": "John Doe",
    "edad": 30,
    "ciudad": "Nueva York",
    "hobbies": [
        "lectura",
        "jardinería",
        "ciclismo"
    ],
    "empleo": {
        "título": "Ingeniero de Software",
        "empresa": "TechCorp"
    }
}
\end{lstlisting}
 
\section{Algoritmo YYY}
\label{sec:algoritmo-yyy}

Este es el clásico entorno \texttt{verbatim}, sin coloreado pero con fuente monoespaciada:

\begin{center}
\begin{footnotesize}
\begin{verbatim}

/***********************************************************************************
 *
 * Fichero .h
 *
 ***********************************************************************************
 *
 * AUTORES
 *
 * FECHA
 *
 * DESCRIPCION
 *
 *
 ************************************************************************************/
 
\end{verbatim}
\end{footnotesize}
\end{center}

\section{Algoritmo ZZZ}
\label{sec:algoritmo-zzz}

Ejemplo de entorno para describir algoritmos en pseudocódigo:

\begin{algorithm}[htpb]
    \caption{Cálculo del factorial de un número}\label{alg:factorial}
    \begin{algorithmic}[1]
        \State \textbf{Entrada:} Un entero $n$
        \State \textbf{Salida:} El factorial de $n$
        \Function{Factorial}{$n$}\Comment{El factorial de n}
            \If{$n \leq 1$}
                \State \Return $1$\Comment{El factorial de 0 o 1 es 1}
            \Else
                \State \Return $n \times$ \Call{Factorial}{n-1}
            \EndIf
        \EndFunction
    \end{algorithmic}
\end{algorithm}

Otra forma de describir algoritmos es utilizar entornos \texttt{lstlisting} y emplear una sintaxis de pseudocódigo similar a alguno de los lenguajes soportados por este paquete, como Python o Pascal.
\chapter{Título del Apéndice 2}
\label{ch:apéndice-dos}

\section{Otro apéndice: Sección 1}
\lipsum[2]

\section{Otro apéndice: Sección 2}
\lipsum[3]

%%%%%%%%%%%%%%%%%%%%%%%%%%%%%%%%%%%%%%%%%%%%%%%%%%%%%%%%%%
% BIBLIOGRAFÍA
%%%%%%%%%%%%%%%%%%%%%%%%%%%%%%%%%%%%%%%%%%%%%%%%%%%%%%%%%%
\backmatter

\printbibliography[prenote=note]

\end{document}